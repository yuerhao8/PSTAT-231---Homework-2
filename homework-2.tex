% Options for packages loaded elsewhere
\PassOptionsToPackage{unicode}{hyperref}
\PassOptionsToPackage{hyphens}{url}
%
\documentclass[
]{article}
\usepackage{amsmath,amssymb}
\usepackage{lmodern}
\usepackage{iftex}
\ifPDFTeX
  \usepackage[T1]{fontenc}
  \usepackage[utf8]{inputenc}
  \usepackage{textcomp} % provide euro and other symbols
\else % if luatex or xetex
  \usepackage{unicode-math}
  \defaultfontfeatures{Scale=MatchLowercase}
  \defaultfontfeatures[\rmfamily]{Ligatures=TeX,Scale=1}
\fi
% Use upquote if available, for straight quotes in verbatim environments
\IfFileExists{upquote.sty}{\usepackage{upquote}}{}
\IfFileExists{microtype.sty}{% use microtype if available
  \usepackage[]{microtype}
  \UseMicrotypeSet[protrusion]{basicmath} % disable protrusion for tt fonts
}{}
\makeatletter
\@ifundefined{KOMAClassName}{% if non-KOMA class
  \IfFileExists{parskip.sty}{%
    \usepackage{parskip}
  }{% else
    \setlength{\parindent}{0pt}
    \setlength{\parskip}{6pt plus 2pt minus 1pt}}
}{% if KOMA class
  \KOMAoptions{parskip=half}}
\makeatother
\usepackage{xcolor}
\usepackage[margin=1in]{geometry}
\usepackage{color}
\usepackage{fancyvrb}
\newcommand{\VerbBar}{|}
\newcommand{\VERB}{\Verb[commandchars=\\\{\}]}
\DefineVerbatimEnvironment{Highlighting}{Verbatim}{commandchars=\\\{\}}
% Add ',fontsize=\small' for more characters per line
\usepackage{framed}
\definecolor{shadecolor}{RGB}{248,248,248}
\newenvironment{Shaded}{\begin{snugshade}}{\end{snugshade}}
\newcommand{\AlertTok}[1]{\textcolor[rgb]{0.94,0.16,0.16}{#1}}
\newcommand{\AnnotationTok}[1]{\textcolor[rgb]{0.56,0.35,0.01}{\textbf{\textit{#1}}}}
\newcommand{\AttributeTok}[1]{\textcolor[rgb]{0.77,0.63,0.00}{#1}}
\newcommand{\BaseNTok}[1]{\textcolor[rgb]{0.00,0.00,0.81}{#1}}
\newcommand{\BuiltInTok}[1]{#1}
\newcommand{\CharTok}[1]{\textcolor[rgb]{0.31,0.60,0.02}{#1}}
\newcommand{\CommentTok}[1]{\textcolor[rgb]{0.56,0.35,0.01}{\textit{#1}}}
\newcommand{\CommentVarTok}[1]{\textcolor[rgb]{0.56,0.35,0.01}{\textbf{\textit{#1}}}}
\newcommand{\ConstantTok}[1]{\textcolor[rgb]{0.00,0.00,0.00}{#1}}
\newcommand{\ControlFlowTok}[1]{\textcolor[rgb]{0.13,0.29,0.53}{\textbf{#1}}}
\newcommand{\DataTypeTok}[1]{\textcolor[rgb]{0.13,0.29,0.53}{#1}}
\newcommand{\DecValTok}[1]{\textcolor[rgb]{0.00,0.00,0.81}{#1}}
\newcommand{\DocumentationTok}[1]{\textcolor[rgb]{0.56,0.35,0.01}{\textbf{\textit{#1}}}}
\newcommand{\ErrorTok}[1]{\textcolor[rgb]{0.64,0.00,0.00}{\textbf{#1}}}
\newcommand{\ExtensionTok}[1]{#1}
\newcommand{\FloatTok}[1]{\textcolor[rgb]{0.00,0.00,0.81}{#1}}
\newcommand{\FunctionTok}[1]{\textcolor[rgb]{0.00,0.00,0.00}{#1}}
\newcommand{\ImportTok}[1]{#1}
\newcommand{\InformationTok}[1]{\textcolor[rgb]{0.56,0.35,0.01}{\textbf{\textit{#1}}}}
\newcommand{\KeywordTok}[1]{\textcolor[rgb]{0.13,0.29,0.53}{\textbf{#1}}}
\newcommand{\NormalTok}[1]{#1}
\newcommand{\OperatorTok}[1]{\textcolor[rgb]{0.81,0.36,0.00}{\textbf{#1}}}
\newcommand{\OtherTok}[1]{\textcolor[rgb]{0.56,0.35,0.01}{#1}}
\newcommand{\PreprocessorTok}[1]{\textcolor[rgb]{0.56,0.35,0.01}{\textit{#1}}}
\newcommand{\RegionMarkerTok}[1]{#1}
\newcommand{\SpecialCharTok}[1]{\textcolor[rgb]{0.00,0.00,0.00}{#1}}
\newcommand{\SpecialStringTok}[1]{\textcolor[rgb]{0.31,0.60,0.02}{#1}}
\newcommand{\StringTok}[1]{\textcolor[rgb]{0.31,0.60,0.02}{#1}}
\newcommand{\VariableTok}[1]{\textcolor[rgb]{0.00,0.00,0.00}{#1}}
\newcommand{\VerbatimStringTok}[1]{\textcolor[rgb]{0.31,0.60,0.02}{#1}}
\newcommand{\WarningTok}[1]{\textcolor[rgb]{0.56,0.35,0.01}{\textbf{\textit{#1}}}}
\usepackage{graphicx}
\makeatletter
\def\maxwidth{\ifdim\Gin@nat@width>\linewidth\linewidth\else\Gin@nat@width\fi}
\def\maxheight{\ifdim\Gin@nat@height>\textheight\textheight\else\Gin@nat@height\fi}
\makeatother
% Scale images if necessary, so that they will not overflow the page
% margins by default, and it is still possible to overwrite the defaults
% using explicit options in \includegraphics[width, height, ...]{}
\setkeys{Gin}{width=\maxwidth,height=\maxheight,keepaspectratio}
% Set default figure placement to htbp
\makeatletter
\def\fps@figure{htbp}
\makeatother
\setlength{\emergencystretch}{3em} % prevent overfull lines
\providecommand{\tightlist}{%
  \setlength{\itemsep}{0pt}\setlength{\parskip}{0pt}}
\setcounter{secnumdepth}{-\maxdimen} % remove section numbering
\ifLuaTeX
  \usepackage{selnolig}  % disable illegal ligatures
\fi
\IfFileExists{bookmark.sty}{\usepackage{bookmark}}{\usepackage{hyperref}}
\IfFileExists{xurl.sty}{\usepackage{xurl}}{} % add URL line breaks if available
\urlstyle{same} % disable monospaced font for URLs
\hypersetup{
  pdftitle={Homework 2},
  pdfauthor={PSTAT 131/231},
  hidelinks,
  pdfcreator={LaTeX via pandoc}}

\title{Homework 2}
\author{PSTAT 131/231}
\date{}

\begin{document}
\maketitle

{
\setcounter{tocdepth}{2}
\tableofcontents
}
\hypertarget{linear-regression}{%
\subsection{Linear Regression}\label{linear-regression}}

For this lab, we will be working with a data set from the UCI
(University of California, Irvine) Machine Learning repository
(\href{http://archive.ics.uci.edu/ml/datasets/Abalone}{see website
here}). The full data set consists of \(4,177\) observations of abalone
in Tasmania. (Fun fact:
\href{https://en.wikipedia.org/wiki/Tasmania}{Tasmania} supplies about
\(25\%\) of the yearly world abalone harvest.)

\begin{figure}
\centering
\includegraphics[width=1.58333in,height=\textheight]{https://cdn.shopify.com/s/files/1/1198/8002/products/1d89434927bffb6fd1786c19c2d921fb_2000x_652a2391-5a0a-4f10-966c-f759dc08635c_1024x1024.jpg?v=1582320404}
\caption{\emph{Fig 1. Inside of an abalone shell.}}
\end{figure}

The age of an abalone is typically determined by cutting the shell open
and counting the number of rings with a microscope. The purpose of this
data set is to determine whether abalone age (\textbf{number of rings +
1.5}) can be accurately predicted using other, easier-to-obtain
information about the abalone.

The full abalone data set is located in the
\texttt{\textbackslash{}data} subdirectory. Read it into \emph{R} using
\texttt{read\_csv()}. Take a moment to read through the codebook
(\texttt{abalone\_codebook.txt}) and familiarize yourself with the
variable definitions.

Make sure you load the \texttt{tidyverse} and \texttt{tidymodels}!

\begin{Shaded}
\begin{Highlighting}[]
\FunctionTok{library}\NormalTok{(tidyverse)}
\FunctionTok{library}\NormalTok{(ggplot2)}
\FunctionTok{library}\NormalTok{(tidymodels)}
\FunctionTok{library}\NormalTok{(corrplot)}
\FunctionTok{library}\NormalTok{(ggthemes)}
\FunctionTok{library}\NormalTok{(yardstick)}
\FunctionTok{tidymodels\_prefer}\NormalTok{() }
\FunctionTok{set.seed}\NormalTok{(}\DecValTok{1000}\NormalTok{)}
\end{Highlighting}
\end{Shaded}

\begin{Shaded}
\begin{Highlighting}[]
\CommentTok{\#load data set}
\FunctionTok{setwd}\NormalTok{(}\StringTok{"/Users/Yuer\_Hao/Desktop/PSTAT 131/PSTAT231 {-} HW2/PSTAT131 {-} homework{-}2/data"}\NormalTok{)}
\NormalTok{abalone\_data }\OtherTok{\textless{}{-}} \FunctionTok{read.csv}\NormalTok{(}\StringTok{"abalone.csv"}\NormalTok{)}
\FunctionTok{head}\NormalTok{(abalone\_data)}
\end{Highlighting}
\end{Shaded}

\begin{verbatim}
##   type longest_shell diameter height whole_weight shucked_weight viscera_weight
## 1    M         0.455    0.365  0.095       0.5140         0.2245         0.1010
## 2    M         0.350    0.265  0.090       0.2255         0.0995         0.0485
## 3    F         0.530    0.420  0.135       0.6770         0.2565         0.1415
## 4    M         0.440    0.365  0.125       0.5160         0.2155         0.1140
## 5    I         0.330    0.255  0.080       0.2050         0.0895         0.0395
## 6    I         0.425    0.300  0.095       0.3515         0.1410         0.0775
##   shell_weight rings
## 1        0.150    15
## 2        0.070     7
## 3        0.210     9
## 4        0.155    10
## 5        0.055     7
## 6        0.120     8
\end{verbatim}

\hypertarget{question-1}{%
\subsubsection{Question 1}\label{question-1}}

Your goal is to predict abalone age, which is calculated as the number
of rings plus 1.5. Notice there currently is no \texttt{age} variable in
the data set. Add \texttt{age} to the data set.

Assess and describe the distribution of \texttt{age}.

\begin{Shaded}
\begin{Highlighting}[]
\CommentTok{\# Add age column to the abalone data set with "rings"+1.5}
\NormalTok{abalone }\OtherTok{\textless{}{-}}\NormalTok{ abalone\_data }\SpecialCharTok{\%\textgreater{}\%}
  \FunctionTok{mutate}\NormalTok{(abalone\_data, }\AttributeTok{age =}\NormalTok{ rings }\SpecialCharTok{+} \FloatTok{1.5}\NormalTok{) }
\FunctionTok{head}\NormalTok{(abalone)}
\end{Highlighting}
\end{Shaded}

\begin{verbatim}
##   type longest_shell diameter height whole_weight shucked_weight viscera_weight
## 1    M         0.455    0.365  0.095       0.5140         0.2245         0.1010
## 2    M         0.350    0.265  0.090       0.2255         0.0995         0.0485
## 3    F         0.530    0.420  0.135       0.6770         0.2565         0.1415
## 4    M         0.440    0.365  0.125       0.5160         0.2155         0.1140
## 5    I         0.330    0.255  0.080       0.2050         0.0895         0.0395
## 6    I         0.425    0.300  0.095       0.3515         0.1410         0.0775
##   shell_weight rings  age
## 1        0.150    15 16.5
## 2        0.070     7  8.5
## 3        0.210     9 10.5
## 4        0.155    10 11.5
## 5        0.055     7  8.5
## 6        0.120     8  9.5
\end{verbatim}

\begin{Shaded}
\begin{Highlighting}[]
\CommentTok{\# Check the distribution of \textasciigrave{}age\textasciigrave{}}
\FunctionTok{ggplot}\NormalTok{(}\AttributeTok{data =}\NormalTok{ abalone, }\FunctionTok{aes}\NormalTok{(age)) }\SpecialCharTok{+}
  \FunctionTok{geom\_histogram}\NormalTok{()}
\end{Highlighting}
\end{Shaded}

\includegraphics{homework-2_files/figure-latex/unnamed-chunk-4-1.pdf}
According to the plot, the distribution of age relatively follows the
normal distribution with mean around 10-12. It is also slightly skewed
to the right. Most of the age data falls between 5 and 18. However,
there exists few of extreme outliers around age 30.

\hypertarget{question-2}{%
\subsubsection{Question 2}\label{question-2}}

Split the abalone data into a training set and a testing set. Use
stratified sampling. You should decide on appropriate percentages for
splitting the data. \emph{Remember that you'll need to set a seed at the
beginning of the document to reproduce your results.}

\begin{Shaded}
\begin{Highlighting}[]
\NormalTok{abalone\_split }\OtherTok{\textless{}{-}} \FunctionTok{initial\_split}\NormalTok{(abalone,}\AttributeTok{prop =} \FloatTok{0.80}\NormalTok{, }\AttributeTok{strata =}\NormalTok{ age)}
\NormalTok{abalone\_train }\OtherTok{\textless{}{-}} \FunctionTok{training}\NormalTok{(abalone\_split)}
\NormalTok{abalone\_test }\OtherTok{\textless{}{-}} \FunctionTok{testing}\NormalTok{(abalone\_split)}
\end{Highlighting}
\end{Shaded}

\hypertarget{question-3}{%
\subsubsection{Question 3}\label{question-3}}

Using the \textbf{training} data, create a recipe predicting the outcome
variable, \texttt{age}, with all other predictor variables. Note that
you should not include \texttt{rings} to predict \texttt{age}. Explain
why you shouldn't use \texttt{rings} to predict \texttt{age}.

Steps for your recipe:

\begin{enumerate}
\def\labelenumi{\arabic{enumi}.}
\item
  dummy code any categorical predictors
\item
  create interactions between

  \begin{itemize}
  \tightlist
  \item
    \texttt{type} and \texttt{shucked\_weight},
  \item
    \texttt{longest\_shell} and \texttt{diameter},
  \item
    \texttt{shucked\_weight} and \texttt{shell\_weight}
  \end{itemize}
\item
  center all predictors, and
\item
  scale all predictors.
\end{enumerate}

You'll need to investigate the \texttt{tidymodels} documentation to find
the appropriate step functions to use.

\begin{Shaded}
\begin{Highlighting}[]
\CommentTok{\#not include \textasciigrave{}rings\textasciigrave{} to predict \textasciigrave{}age\textasciigrave{}}
\NormalTok{aba\_train\_no\_rings }\OtherTok{\textless{}{-}}\NormalTok{ abalone\_train }\SpecialCharTok{\%\textgreater{}\%}
  \FunctionTok{select}\NormalTok{(}\SpecialCharTok{{-}}\NormalTok{rings)}

\CommentTok{\#create a recipe}
\NormalTok{abalone\_recipe }\OtherTok{\textless{}{-}} \FunctionTok{recipe}\NormalTok{(age }\SpecialCharTok{\textasciitilde{}}\NormalTok{ ., }\AttributeTok{data =}\NormalTok{ aba\_train\_no\_rings) }\SpecialCharTok{\%\textgreater{}\%}
  \FunctionTok{step\_dummy}\NormalTok{(}\FunctionTok{all\_nominal\_predictors}\NormalTok{()) }\SpecialCharTok{\%\textgreater{}\%}
  \FunctionTok{step\_interact}\NormalTok{(}\AttributeTok{terms =} \SpecialCharTok{\textasciitilde{}} \FunctionTok{starts\_with}\NormalTok{(}\StringTok{"type"}\NormalTok{)}\SpecialCharTok{:}\NormalTok{shucked\_weight }\SpecialCharTok{+}
\NormalTok{                  longest\_shell}\SpecialCharTok{:}\NormalTok{diameter }\SpecialCharTok{+}
\NormalTok{                  shucked\_weight}\SpecialCharTok{:}\NormalTok{shell\_weight) }\SpecialCharTok{\%\textgreater{}\%}
  \FunctionTok{step\_center}\NormalTok{(}\FunctionTok{all\_predictors}\NormalTok{()) }\SpecialCharTok{\%\textgreater{}\%}
  \FunctionTok{step\_scale}\NormalTok{(}\FunctionTok{all\_predictors}\NormalTok{())}
\end{Highlighting}
\end{Shaded}

Data rings cannot use to predict age since the age column is just the
linear transformation (age = rings + 1.5) of the rings column.
Therefore, they have exactly the same distribution and trend with shift.

\hypertarget{question-4}{%
\subsubsection{Question 4}\label{question-4}}

Create and store a linear regression object using the \texttt{"lm"}
engine.

\begin{Shaded}
\begin{Highlighting}[]
\NormalTok{lm\_model }\OtherTok{=} \FunctionTok{linear\_reg}\NormalTok{() }\SpecialCharTok{\%\textgreater{}\%}
  \FunctionTok{set\_engine}\NormalTok{(}\StringTok{"lm"}\NormalTok{)}
\end{Highlighting}
\end{Shaded}

\hypertarget{question-5}{%
\subsubsection{Question 5}\label{question-5}}

Now:

\begin{enumerate}
\def\labelenumi{\arabic{enumi}.}
\tightlist
\item
  set up an empty workflow,
\item
  add the model you created in Question 4, and
\item
  add the recipe that you created in Question 3.
\end{enumerate}

\begin{Shaded}
\begin{Highlighting}[]
\NormalTok{lm\_wkflow }\OtherTok{\textless{}{-}} \FunctionTok{workflow}\NormalTok{() }\SpecialCharTok{\%\textgreater{}\%}
  \FunctionTok{add\_model}\NormalTok{(lm\_model) }\SpecialCharTok{\%\textgreater{}\%}
  \FunctionTok{add\_recipe}\NormalTok{(abalone\_recipe)}
\end{Highlighting}
\end{Shaded}

\hypertarget{question-6}{%
\subsubsection{Question 6}\label{question-6}}

Use your \texttt{fit()} object to predict the age of a hypothetical
female abalone with longest\_shell = 0.50, diameter = 0.10, height =
0.30, whole\_weight = 4, shucked\_weight = 1, viscera\_weight = 2,
shell\_weight = 1.

\begin{Shaded}
\begin{Highlighting}[]
\NormalTok{lm\_fit }\OtherTok{\textless{}{-}} \FunctionTok{fit}\NormalTok{(lm\_wkflow, aba\_train\_no\_rings)}
\NormalTok{female\_aba\_pred }\OtherTok{\textless{}{-}} \FunctionTok{data.frame}\NormalTok{(}\AttributeTok{type =} \StringTok{"F"}\NormalTok{, }
                              \AttributeTok{longest\_shell =} \FloatTok{0.50}\NormalTok{, }
                              \AttributeTok{diameter =} \FloatTok{0.10}\NormalTok{, }
                              \AttributeTok{height =} \FloatTok{0.30}\NormalTok{, }
                              \AttributeTok{whole\_weight =} \DecValTok{4}\NormalTok{, }
                              \AttributeTok{shucked\_weight =} \DecValTok{1}\NormalTok{, }
                              \AttributeTok{viscera\_weight =} \DecValTok{2}\NormalTok{, }
                              \AttributeTok{shell\_weight =} \DecValTok{1}\NormalTok{)}

\NormalTok{lm\_fit }\SpecialCharTok{\%\textgreater{}\%} 
  \CommentTok{\# This returns the parsnip object:}
  \FunctionTok{extract\_fit\_parsnip}\NormalTok{() }\SpecialCharTok{\%\textgreater{}\%} 
  \CommentTok{\# Now tidy the linear model object:}
  \FunctionTok{tidy}\NormalTok{()}
\end{Highlighting}
\end{Shaded}

\begin{verbatim}
## # A tibble: 14 x 5
##    term                          estimate std.error statistic  p.value
##    <chr>                            <dbl>     <dbl>     <dbl>    <dbl>
##  1 (Intercept)                     11.4      0.0373   307.    0       
##  2 longest_shell                    0.312    0.290      1.08  2.81e- 1
##  3 diameter                         2.29     0.319      7.17  8.98e-13
##  4 height                           0.204    0.0699     2.92  3.52e- 3
##  5 whole_weight                     4.84     0.401     12.1   8.62e-33
##  6 shucked_weight                  -4.22     0.255    -16.5   5.29e-59
##  7 viscera_weight                  -0.928    0.160     -5.81  6.85e- 9
##  8 shell_weight                     1.76     0.219      8.04  1.21e-15
##  9 type_I                          -0.909    0.114     -7.98  2.01e-15
## 10 type_M                          -0.243    0.103     -2.36  1.82e- 2
## 11 type_I_x_shucked_weight          0.462    0.0857     5.39  7.55e- 8
## 12 type_M_x_shucked_weight          0.260    0.108      2.41  1.61e- 2
## 13 longest_shell_x_diameter        -2.78     0.399     -6.96  4.01e-12
## 14 shucked_weight_x_shell_weight   -0.164    0.206     -0.793 4.28e- 1
\end{verbatim}

\begin{Shaded}
\begin{Highlighting}[]
\FunctionTok{predict}\NormalTok{(lm\_fit, }\AttributeTok{new\_data =}\NormalTok{ female\_aba\_pred)}
\end{Highlighting}
\end{Shaded}

\begin{verbatim}
## # A tibble: 1 x 1
##   .pred
##   <dbl>
## 1  22.7
\end{verbatim}

\hypertarget{question-7}{%
\subsubsection{Question 7}\label{question-7}}

Now you want to assess your model's performance. To do this, use the
\texttt{yardstick} package:

\begin{enumerate}
\def\labelenumi{\arabic{enumi}.}
\tightlist
\item
  Create a metric set that includes \emph{R\textsuperscript{2}}, RMSE
  (root mean squared error), and MAE (mean absolute error).
\item
  Use \texttt{predict()} and \texttt{bind\_cols()} to create a tibble of
  your model's predicted values from the \textbf{training data} along
  with the actual observed ages (these are needed to assess your model's
  performance).
\item
  Finally, apply your metric set to the tibble, report the results, and
  interpret the \emph{R\textsuperscript{2}} value.
\end{enumerate}

\begin{Shaded}
\begin{Highlighting}[]
\NormalTok{abalone\_train\_rlt }\OtherTok{\textless{}{-}} \FunctionTok{predict}\NormalTok{(lm\_fit, }\AttributeTok{new\_data =}\NormalTok{ aba\_train\_no\_rings }\SpecialCharTok{\%\textgreater{}\%} \FunctionTok{select}\NormalTok{(}\SpecialCharTok{{-}}\NormalTok{age)) }
\NormalTok{abalone\_train\_rlt }\OtherTok{\textless{}{-}} \FunctionTok{bind\_cols}\NormalTok{(abalone\_train\_rlt, aba\_train\_no\_rings }\SpecialCharTok{\%\textgreater{}\%} \FunctionTok{select}\NormalTok{(age))}

\FunctionTok{head}\NormalTok{(abalone\_train\_rlt)                        }
\end{Highlighting}
\end{Shaded}

\begin{verbatim}
## # A tibble: 6 x 2
##   .pred   age
##   <dbl> <dbl>
## 1  8.10   8.5
## 2  9.33   9.5
## 3 10.5    8.5
## 4 10.1    9.5
## 5 11.0    9.5
## 6  6.35   6.5
\end{verbatim}

\begin{Shaded}
\begin{Highlighting}[]
\NormalTok{abalone\_metrics}\OtherTok{\textless{}{-}}\FunctionTok{metric\_set}\NormalTok{(rmse,rsq,mae)}
\FunctionTok{abalone\_metrics}\NormalTok{(abalone\_train\_rlt, }\AttributeTok{truth=}\NormalTok{age,}
                \AttributeTok{estimate=}\NormalTok{.pred)}
\end{Highlighting}
\end{Shaded}

\begin{verbatim}
## # A tibble: 3 x 3
##   .metric .estimator .estimate
##   <chr>   <chr>          <dbl>
## 1 rmse    standard       2.15 
## 2 rsq     standard       0.562
## 3 mae     standard       1.55
\end{verbatim}

After applying metric set to the tibble, the results shows the value of
\emph{R\textsuperscript{2}} value is 0.5618608 approximately which
indicates that 56.18608\% of the data fit the regression model.

\hypertarget{required-for-231-students}{%
\subsubsection{Required for 231
Students}\label{required-for-231-students}}

In lecture, we presented the general bias-variance tradeoff, which takes
the form:

\[
E[(y_0 - \hat{f}(x_0))^2]=Var(\hat{f}(x_0))+[Bias(\hat{f}(x_0))]^2+Var(\epsilon)
\]

where the underlying model \(Y=f(X)+\epsilon\) satisfies the following:

\begin{itemize}
\tightlist
\item
  \(\epsilon\) is a zero-mean random noise term and \(X\) is non-random
  (all randomness in \(Y\) comes from \(\epsilon\));
\item
  \((x_0, y_0)\) represents a test observation, independent of the
  training set, drawn from the same model;
\item
  \(\hat{f}(.)\) is the estimate of \(f\) obtained from the training
  set.
\end{itemize}

\hypertarget{question-8}{%
\paragraph{Question 8}\label{question-8}}

Which term(s) in the bias-variance tradeoff above represent the
reproducible error? Which term(s) represent the irreducible error?

\hypertarget{question-9}{%
\paragraph{Question 9}\label{question-9}}

Using the bias-variance tradeoff above, demonstrate that the expected
test error is always at least as large as the irreducible error.

\hypertarget{question-10}{%
\paragraph{Question 10}\label{question-10}}

Prove the bias-variance tradeoff.

Hints:

\begin{itemize}
\tightlist
\item
  use the definition of \(Bias(\hat{f}(x_0))=E[\hat{f}(x_0)]-f(x_0)\);
\item
  reorganize terms in the expected test error by adding and subtracting
  \(E[\hat{f}(x_0)]\)
\end{itemize}

\end{document}
